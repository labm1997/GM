\documentclass{article}
\usepackage[portuguese]{babel}
\usepackage[utf8]{inputenc}
\usepackage[top=2cm,bottom=2cm,left=2cm,right=2cm]{geometry}
\usepackage{enumerate}
\usepackage[T1]{fontenc}
\usepackage[dvipsnames]{xcolor}
\usepackage{color}
\usepackage{longtable}
\usepackage{amssymb}

\newcommand{\pendente}{\textbf{...}}
\newcommand{\concluido}{\checkmark}
\newcommand{\atrasado}{\color{OrangeRed} $\times$}

\newenvironment{tarefas}{\begin{longtable}{llr}}{\end{longtable}}

\newcommand{\tarefa}[5]{
  \multicolumn{3}{l}{\parbox{15cm}{\textbf{Tarefa} \emph{#1}\\ #2}}
	\\\\\hline\textbf{Prazo e responsáveis:} & #3, #4  & #5\\\hline\\
}

\title{Planejamento}
\date{}

\begin{document}
  \maketitle
  \begin{tarefas}
	  \tarefa{Configurar servidor}{
	    Configurar o servidor para uso de todos os integrantes do grupo. Um servidor mariadb (\texttt{oluiz.com.br:3306}), um repositório no GitHub e o servidor que rodará o programa (\texttt{oluiz.com.br:9000}). Configurar o projeto usando \texttt{play framework}
	  }{23 de Outubro}{Luiz Antônio}{\concluido}
	  \tarefa{Construir página inicial do GM}{
		  Criar um formulário de login do tipo POST com identificação do usuário e senha. Criar uma logomarca para o GM.\\
		  \emph{Restrição}\\
		  Deve ser usado CSS, HTML e JavaScript. Pode usar \textit{bootstrap}.
	  }{27 de Outubro}{Ivan e Hélio}{\pendente}
	  \tarefa{Construir páginas}{
	    Construir as páginas: oferta (Campus $\rightarrow$ Instituto $\rightarrow$ Departamento $\rightarrow$ Lista de matérias), curso: (Campus $\rightarrow$ Instituto $\rightarrow$ Departamento $\rightarrow$ Lista de cursos), página do curso, matéria, grade horária, resultado, remoção de matrícula, trancamento, oferta baseada no fluxo, confirmação da lista de espera, trancamento geral, dados pessoais e histórico escolar.
		  Tomando como base o MatriculaWeb. Caso deseje mudar a estrutura das páginas para algo mais conveniente sinta-se livre.\\
		  \emph{Restrição}
		  Deve ser usado CSS, HTML e JavaScript. Pode usar \textit{bootstrap}.\\
		  \emph{Detalhes}
		  Cada página deve ter \texttt{<input type=\textquotedbl text\textquotedbl></input>} caso sejam dados modificáveis, por exemplo: endereço residencial, telefone na página de dados pessoais e botões \texttt{<input type=\textquotedbl button\textquotedbl></input>} para formulários etc. Caso ache melhor, por questões de design, usar \texttt{<div>...</div>} sinta-se à vontade.
	  }{27 de Outubro}{Ivan e Hélio}{\pendente}
	  \tarefa{Definir tabelas}{
	    Construir e definir as tabelas do banco de dados. 
	  }{27 de Outubro}{Luiz Antônio e Vinícius}{\pendente}
	  \tarefa{Definir case classes}{
	    Definir e implementar as \texttt{case classes} a serem usadas no servidor scala.
	  }{03 de Novembro}{Luiz Antônio e Vinícius}{\pendente}
	\end{tarefas}
\end{document}
